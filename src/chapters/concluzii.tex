\chapter{Concluzii}

Securitatea informa\c{t}iilor este un subiect sensibil care este \^{i}n continu\u{a} mi\c{s}care, deoarece de-a lungul timpului, atacatorii g\u{a}sesc metode noi de a produce pagube sau de a \^{i}nc\u{a}lca dreptul la intimitate a persoanelor ce utilizeaz\u{a} mediul virtual.

\^{I}n prezent, foarte multe persoane folosesc re\c{t}elele de socializare pentru a interac\c{t}iona \^{i}n spa\c{t}iul virtual. Faptul c\u{a} num\u{a}rul de utilizatori este mare reprezint\u{a} o atrac\c{t}ie \c{s}i o provocare pentru atacatori.

Scopul proiectului a fost de a \^{i}mbun\u{a}t\u{a}\c{t}ii securitatea interac\c{t}iunilor din cadrul unei re\c{t}ele sociale, precum Facebook, acest lucru realiz\^{a}ndu-se printr-o aplica\c{t}ie pentru dispozitivele mobile care folosesc Android ca sistem de operare. De\c{s}i implementarea actual\u{a} nu ofer\u{a} toate func\c{t}ionalit\u{a}\c{t}ile puse la dispozi\c{t}ie de Facebook, aceasta poate fi dezvoltat\u{a} \^{i}n viitor prin ad\u{a}ugarea treptat\u{a} de noi func\c{t}ionalit\u{a}\c{t}i.

Securitatea interac\c{t}iunilor a fost \^{i}mbun\u{a}t\u{a}\c{t}it\u{a} cript\^{a}nd datele transmise prin intermediul platformei sociale. Obiectivele principale au fost de a proteja mesajele din conversa\c{t}iile private \c{s}i cele din cadrul grupurilor de Facebook. La nivel de func\c{t}ionalitate, aceste obiective au fost atinse.

Solu\c{t}ia de criptare aleas\u{a} nu necesit\u{a} resurse multe, procesul fiind destul de scurt din punct de vedere al timpului. Astfel, faptul c\u{a} mesajele ce sunt gestionate de aplica\c{t}ie sunt criptate \c{s}i decriptate nu este un lucru sesizabil pentru utilizator. \^{I}n plus, pentru implementarea acestui proiect s-au folosit biblioteci \c{s}i module disponibile \^{i}n Android care ajut\u{a} la reducerea timpului de reac\c{t}ie a interfe\c{t}ei grafice la ac\c{t}iunile realizate de utilizator.

Mesajele sunt stocate pe platforma social\u{a} \^{i}n forma lor criptat\u{a}, dac\u{a} se dore\c{s}te citirea lor este nevoie de cheia privat\u{a} folosit\u{a} la criptare, deci un atacator nu ar trebuie s\u{a} poat\u{a} decripta mesajele.

\^{I}n concluzie, scopul lucr\u{a}rii de a oferi o aplica\c{t}ie pentru interac\c{t}ionarea protejat\u{a} \^{i}n cadrul unei re\c{t}ele de socializare a fost atins. 

\section{Activit\u{a}\c{t}i viitoare}

O activitate viitoare destul de important\u{a} ar fi \^{i}mbun\u{a}t\u{a}\c{t}irea serviciului care se ocup\u{a} de gestionarea notific\u{a}rilor. Acesta ar trebui s\u{a} ruleze tot timpul c\^{a}t aplica\c{t}ia este pornit\u{a}, iar \^{i}n momentul \^{i}n care se deschide activitatea folosit\u{a} pentru schimbul de mesaje s\u{a} \^{i}n\c{s}tiin\c{t}eze activitatea \^{i}n cazul \^{i}n care s-au primit notific\u{a}ri \^{i}nainte de deschiderea acesteia.

O alt\u{a} activitate presupune implementarea unei solu\c{t}ii de trimitere automat\u{a} a cheilor private prin intermediul unor SMS-uri. Acest lucru implic\u{a} aflarea num\u{a}rului de telefon al utilizatorilor.

La partea de schimb de mesaje se va desc\u{a}rca, afi\c{s}a \c{s}i decripta mesajele trimise p\^{a}n\u{a} \^{i}n acel moment.

Pentru partea de grupuri, se va \^{i}ncerca g\u{a}sirea unor solu\c{t}ii de a posta imagini criptate \^{i}n cadrul acestora, iar post\u{a}rile s\u{a} aib\u{a} drept autor numele utilizatorului, nu cel al aplica\c{t}iei. De asemenea se va \^{i}ncerca introducerea de noi func\c{t}ionalit\u{a}\c{t}i, precum afi\c{s}area comentariilor de la post\u{a}ri sau a unei liste cu membrii grupurilor.

\^{I}n plus, se va modifica interfa\c{t}a grafic\u{a} pentru ca utilizarea aplica\c{t}iei s\u{a} fie c\^{a}t mai u\c{s}oar\u{a} \c{s}i intuitiv\u{a} celor care o folosesc. 