\chapter{Introducere}
\label{chapter:intro}

\textbf{This is just a demo file. It should not be used as a sample for a thesis.}\\
\todo{Remove this line (this is a TODO)}

\section{Scopul proiectului}
\label{sub-sec:proj-scope}

This thesis presents the \textbf{\project}.

This is an example of a footnote \footnote{\url{www.google.com}}. You can see here a reference to \labelindexref{Section}{sub-sec:proj-objectives}.

Here we have defined the CS abbreviation.\abbrev{CS}{Computer Science} and the UPB abbreviation.\abbrev{UPB}{University Politehnica of Bucharest}

The main scope of this project is to qualify xLuna for use in critical systems.


Lorem ipsum dolor sit amet, consectetur adipiscing elit. Aenean aliquam lectus vel orci malesuada accumsan. Sed lacinia egestas tortor, eget tristiqu dolor congue sit amet. Curabitur ut nisl a nisi consequat mollis sit amet quis nisl. Vestibulum hendrerit velit at odio sodales pretium. Nam quis tortor sed ante varius sodales. Etiam lacus arcu, placerat sed laoreet a, facilisis sed nunc. Nam gravida fringilla ligula, eu congue lorem feugiat eu.

Lorem ipsum dolor sit amet, consectetur adipiscing elit. Aenean aliquam lectus vel orci malesuada accumsan. Sed lacinia egestas tortor, eget tristiqu dolor congue sit amet. Curabitur ut nisl a nisi consequat mollis sit amet quis nisl. Vestibulum hendrerit velit at odio sodales pretium. Nam quis tortor sed ante varius sodales. Etiam lacus arcu, placerat sed laoreet a, facilisis sed nunc. Nam gravida fringilla ligula, eu congue lorem feugiat eu.


\section{Obiectivele proiectului}
\label{sub-sec:proj-objectives}

We have now included \labelindexref{Figure}{img:report-framework}.




Lorem ipsum dolor sit amet, consectetur adipiscing elit. Aenean aliquam lectus vel orci malesuada accumsan. Sed lacinia egestas tortor, eget tristiqu dolor congue sit amet. Curabitur ut nisl a nisi consequat mollis sit amet quis nisl. Vestibulum hendrerit velit at odio sodales pretium. Nam quis tortor sed ante varius sodales. Etiam lacus arcu, placerat sed laoreet a, facilisis sed nunc. Nam gravida fringilla ligula, eu congue lorem feugiat eu.

We can also have citations like \cite{iso-odf}.
