\chapter{Introducere}
\label{chapter:intro}

\^{I}n aceast\u{a} lucrare de licen\c{t}\u{a} se va discuta despre securizarea interac\c{t}iunilor online, accentul fiind pus pe comunicarea din cadrul unei re\c{t}ele de socializare. Securitatea datelor din mediul virtual a devenit un subiect tot mai important abordat de foarte mult\u{a} lume.

Re\c{t}elele sociale de\c{t}in foarte multe informa\c{t}ii personale despre utilizatorii lor, de aceea au ap\u{a}rut atacuri asupra re\c{t}elelor de socializare. Acestea au ca scop aflarea de informa\c{t}ii private despre o persoan\u{a}.   

Av\^{a}nd \^{i}n vedere faptul c\u{a} dispozitivele mobile au devenit foarte r\u{a}sp\^{a}ndite, acest proiect dore\c{s}te s\u{a} aduc\u{a} un plus de securitate comunica\c{t}iei ini\c{t}iate de pe dizpozitivele care folosesc Android ca sistem de operare. Re\c{t}eaua de socializare aleas\u{a} este Facebook.  

\section{Motiva\c{t}ia proiectului}
\label{sub-sec:proj-scope}

\^{I}n momentul actual re\c{t}elele de socializare nu pun la dispozi\c{t}ie solu\c{t}ii de criptare a datelor postate de membrii lor. Astfel, dac\u{a} un atacator reu\c{s}e\c{s}te s\u{a} ob\c{t}in\u{a} creden\c{t}ialele unui utilizator, acesta poate accesa toate informa\c{t}iile private ale victimei far\u{a} nicio problem\u{a}. De accea, criptarea datelor pune \^{i}n dificultate atacatorul.

Pe l\^{a}ng\u{a} faptul c\u{a} un atacator nu poate descifra datele, criptarea se face doar pe dispozitivele utilizatorilor, astfel echipamentele re\c{t}elelor sociale nu ar fi solicitate \^{i}n plus pentru procesul de criptare sau decriptare a datelor.

Dispozitivele mobile au evoluat foarte mult \^{i}n ultimii ani, acestea beneficiind de procesoare mult mai puternice, astfel procesul de criptare sau decriptare \^{i}n timp real a unor date de dimensiuni mici, nu reprezint\u{a} o problem\u{a}.

Un atacator poate \^{i}ncerca s\u{a} se foloseasc\u{a} de faptul c\u{a} dispozitivele mobile folosesc medii de transmitere f\u{a}r\u{a} fir. Acestea sunt detul de nesigure \^{i}n momentul actual, iar un atacator poate g\u{a}si o modalitate de a intercepta transmisiile de date. Astfel dac\u{a} datele sunt transmise \c{s}i recep\c{t}ionate sub form\u{a} criptat\u{a}, cel care intercepteaz\u{a} comunica\c{t}iile nu are cum s\u{a} descifreze informa\c{t}iile.

Am ales sistemul de operare Android pentru c\u{a} este un proiect open-source \^{i}n continu\u{a} dezvoltare care c\^{a}\c{s}tig\u{a} din ce \^{i}n ce mai mul\c{t}i utilizatori, deci este o \c{t}int\u{a} a atacurilor.

Facebook, a fost ales deoarece este printre cele mai mari re\c{t}ele de socializare la momentul scrierii acestei lucr\u{a}ri. De obicei, platformele cele mai populare devin \c{t}inta unui num\u{a}r mare de atacuri.

\section{Obiectivele proiectului}
\label{sub-sec:proj-objectives}

Principalele obiective ale acestei lucr\u{a}ri de licen\c{t}\u{a} sunt: 

\begin{enumerate}
\item Crearea unei aplica\c{t}ii Android care s\u{a} ofere posibilitatea interac\c{t}ion\u{a}rii cu re\c{t}eaua de socializare Facebook
\item Schimbul de mesaje \^{i}n cadrul re\c{t}elei
\item Gestionarea grupurilor protejate \c{s}i postarea de mesaje \^{i}n cadrul acestora
\item Criptarea datelor postate
\item Decriptarea datelor postate din aceast\u{a} aplica\c{t}ie
\end{enumerate}

