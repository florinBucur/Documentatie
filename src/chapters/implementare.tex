\chapter{Implementare}

\^{I}n acest capitol se vor prezenta detalii despre modul de implementare a func\c{t}ionalit\u{a}\c{t}ilor aplica\c{t}iei. Se vor oferii am\u{a}nunte legate de interfa\c{t}a grafic\u{a}, procesul de autentificare, gestionare a grupurilor, schimbul de mesaje private \c{s}i criptarea, respectiv decriptarea datelor transmise. 

Pe scurt, aplica\c{t}ia ofer\u{a} posibilitatea autentific\u{a}rii pe baza datelor de acreditare a unui cont pe re\c{t}eaua de socializare Facebook \c{s}i a folosirii anumitor func\c{t}ionalit\u{a}\c{t}i puse la dispozi\c{t}ie de Graph API. Gestionarea grupurilor se face cu ajutorul metodelor oferite de platforma Facebook, iar schimbul de mesaje, respectiv criptarea si decriptarea lor nu au suport pentru dezvoltarea de aplica\c{t}ii Android. Pentru partea de mesaje private este folosit un serviciu de gestionare a notific\u{a}rilor primite \^{i}n sistemul de operare Android. Criptarea datelor este de tip AES \c{s}i folose\c{s}te chei simetrice.  

\section{Implementarea aplica\c{t}iei}
\subsection{Interfa\c{t}\u{a} grafic\u{a}}

Dezvoltarea de aplica\c{t}ii Android este foarte flexibil\u{a} put\^{a}nd implementa un lucru folosind mai multe metode. Flexibilitatea se poate observa \^{i}n faptul c\u{a} partea de interfa\c{t}\u{a} grafic\u{a} poate fi separat\u{a} de partea de cod care implementeaz\u{a} functionalit\u{a}\c{t}ile interfe\c{t}ei grafice. Astfel pentru a dezvolta interfa\c{t}a grafic\u{a} se creaz\u{a} fi\c{s}iere de tip XML \^{i}n care se definesc elementele grafice, c\^{a}t \c{s}i propriet\u{a}\c{t}ile lor, cum ar fi culoare, dimensiune sau pozi\c{t}ionarea sa pe ecran.

	Pentru aceast\u{a} aplica\c{t}ie am creat 9 fi\c{s}iere XML, reprezent\^{a}nd layout-urile interfe\c{t}ei grafice: login_fragment.xml, activity_main.xml, activity_friends.xml, activity_groups.xml, chat_activity.xml, friend_row.xml,  group_row.xml, chat_row.xml \c{s}i message_row.xml. 
	
	Primul dintre ele este login_fragment.xml, care este folosit pentru fereastra de autentificare \^{i}n aplica\c{t}ie. Acesta este format dintr-un buton de autentificare, buton pus la dispozi\c{t}ie de API-ul de Facebook, un element de tip TextView folosit pentru a afi\c{s}a un mesaj de \^{i}nt\^{a}mpinare dupa ce procesul de autentificare se termin\u{a} cu succes \c{s}i un obiect de tip ProfilePictureView, care este oferit tot de API-ul de Facebook, afi\c{s}\^{a}nd imaginea de profil a utilizatorului proasp\u{a} autentificat. Acestea sunt parte a unui LinearLayout cu orientarea vertical\u{a}.

	Urm\u{a}torul este activity_main.xml, \^{i}n care este inclus fragmentul de autentificare. Con\c{t}ine un framelayout care ajut\u{a} la definirea \c{s}i includerea fragmentului de autentificare \^{i}n interfa\c{t}a grafic\u{a}.
	
	Urmeaz\u{a} activity_friends.xml, care con\c{t}ine doar un element de tip RecyclerView \c{s}i este folosit pentru a afi\c{s}a pe ecran o list\u{a} cu to\c{t}i prietenii care folosesc aceast\u{a} aplica\c{t}ie. Un r\^{a}nd din list\u{a} este reprezentat printr-un layout compus dintr-un element de tip ProfilePictureView, care afi\c{s}eaz\u{a} imaginea de profil al prietenului \c{s}i un element de tip TextView folosit pentru afi\c{s}area numelui prietenului. Prin selectarea unui prieten din aceast\u{a} list\u{a} se deschide o nou\u{a} activitate, care este reprezentat\u{a} prin fi\c{s}ierul chat_activity.xml. 
	
	Layout-ul chat_activity.xml este compus dintr-o parte folosit\u{a} pentru chat \c{s}i o parte folosit\u{a} pentru gestionarea de grupuri. Partea de chat con\c{t}ine un element ProfilePictureView pentru imaginea de profil al prietenului, un buton, care prin ap\u{a}sarea lui se trimite r\u{a}spunsul completat \^{i}n EditText, persoanei de la care a venit mesajul \c{s}i un buton "Initiate conversation", care prin ap\u{a}sarea lui se deschide aplica\c{t}ia nativ\u{a} de Facebook Messenger. Astfel se poate ini\c{t}ia o conversa\c{t}ie.

Fereastra de chat este reprezentat\u{a} \c{s}i ea printr-un obiect RecyclerView, mesajele fiind afi\c{s}ate separat \c{s}i \^{i}n ordinea \^{i}n care au fost create.

\fig[scale=0.55]{src/img/UI.pdf}{img:Interface}{Interfa\c{t}\u{a} grafic\u{a}}	
\newpage
	Partea de grupuri are \^{i}n componen\c{t}\u{a} un buton "Create group", care are asociat un camp de tip EditText. \^{I}n momentul \^{i}n care se apas\u{a} butonul se preia textul din c\^{a}mpul text \c{s}i se creaz\u{a} un nou grup al aplica\c{t}iei care va avea numele reg\u{a}sit \^{i}n EditText-ul asociat. De asemenea \c{s}i aici exist\u{a} un element de tip RecyclerView, folosit pentru afi\c{s}area unei liste de grupuri existente, care se poate actualiza \^{i}n func\c{t}ie de opera\c{t}iile efectuate, de exemplu, \^{i}n momentul \^{i}n care se creaz\u{a} un grup nou, lista de grupui se actualizeaz\u{a} imediat ce opera\c{t}ia s-a terminat cu succes. 
	
	Fiecare r\^{a}nd din obiectul RecyclerView folosit pentru lista de grupuri este reprezentat prin trei elemente de grafic\u{a}. Un element de tip CheckBox, folosit pentru selectarea grupului asupra c\u{a}ruia urmeaz\u{a} s\u{a} se efectueze o ac\c{t}iune, un TextView pentru afi\c{s}area numelui grupului \c{s}i un buton pe care dac\u{a} se apas\u{a} se va deschide o nou\u{a} fereastr\u{a} \^{i}n care se reg\u{a}sesc post\u{a}rile din grupul selectat.
	
	 Mai sunt prezente butoane pentru ad\u{a}gare utilizator \^{i}n grupul selectat sau \c{s}tergere din grupul selectat. Persoana care este ad\u{a}ugat\u{a} sau eliminat\u{a} din grup este cea selectat\u{a} din lista de prieteni prezentat\u{a} anterior. Butonul "Delete group" este folosit pentru \c{s}tergerea \^{i}n \^{i}ntregime a unui grup.  
	
Layout-ul activity_groups.xml con\c{t}ine un obiect RecyclerView pentru a afi\c{s}a lista post\u{a}rilor din grupul selectat. Tot aici se reg\u{a}se\c{s}te \c{s}i un buton care are asociat un c\^{a}mp text, folosit pentru a posta un nou mesaj.	
	
	Pentru implementarea elementelor de tip RecyclerView este necesar\u{a} crearea de mai multe clase. O clas\u{a} care va reprezenta un obiect de tip Adapter ce se va ocupa de ini\c{t}ializarea listei din interfa\c{t}a grafic\u{a}. La crearea acestui obiect trebuie pasate ca argumente o list\u{a} care con\c{t}ine datele ce vor fi afi\c{s}ate \c{s}i contextul \^{i}n care vor fi afi\c{s}ate. De asemenea, ascult\u{a}torii de tip ap\u{a}sare pe un r\^{a}nd al listei sau ap\u{a}sare a unui anumit element dintr-un r\^{a}nd, sunt definite tot \^{i}n aceast\u{a} clas\u{a}.
	
	 O alt\u{a} clas\u{a} reprezint\u{a} un obiect de tip ViewHolder, care va con\c{t}ine elementele grafice folosite. Un r\^{a}nd din list\u{a} reprezint\u{a} un obiect care cont\c{t}ine informa\c{t}iile ce vor fi completate pe acel r\^{a}nd. 

\subsection{Autentificare}

Modulul de autentificare este reprezentat printr-o clas\u{a} Java, FacebookLoginFragment, care extinde clasa Fragment. Procesul de autentificare presupune, completarea c\u{a}mpurilor user \c{s}i parol\u{a} din interfa\c{t}a grafic\u{a} pus\u{a} la dispozi\c{t}ie de Facebook SDK prin butonul de autentificare. \^{I}n cazul \^{i}n care autentificarea a fost efectuat\u{a} anterior, numele de utilizator \c{s}i parola acestuia sunt preluate din sistemul de operare, mai exact din sec\c{t}iunea "Accounts".

	Dac\u{a} datele de autentificare sunt corecte, se va ob\c{t}ine un token de acces valid, c\u{a}ruia \^{i}i sunt asociate \c{s}i permisiunile necesare utiliz\u{a}rii normale a aplica\c{t}iei. Cum am spus \c{s}i \^{i}n sec\c{t}iunea \ref{facebookSDK}, pentru ca procesul de autentificare s\u{a} func\c{t}ioneze, trebuie s\u{a} avem creat\u{a} o aplica\c{t}ie pe platforma Facebook, inclus Facebook SDK \^{i}n proiect, generat identificatorul aplica\c{t}iei \c{s}i ad\u{a}ugat \^{i}n codul aplica\c{t}iei \c{s}i generat\u{a} o cheie pentru profilul dezvoltatorului.
	
	Autentificarea \^{i}ncepe \^{i}n momentul \^{i}n care se apas\u{a} butonul din interfa\c{t}a grafic\u{a}. Butonul acesta se ocup\u{a} \c{s}i de permisiuni, dar urm\u{a}re\c{s}te \c{s}i starea \^{i}n care se afl\u{a} procesul. Astfel la terminarea procesului, se afi\c{s}eaz\u{a} un mesaj \^{i}n fun\c{t}ie de rezultatul autentific\u{a}rii.
	
	\^{I}n metoda onCreate a fragmentului trebuie s\u{a} se ini\c{t}ializeze Facebook SDK \c{s}i un obiect de tip CallbackManager folosit pentru gestionarea opera\c{t}iilor de tip FacebookCallback. De asemenea, tot aici, datorit\u{a} faptului c\u{a} token-urile de acces nu sunt permanente, pot expira \^{i}n orice moment, se verific\u{a} \c{s}i validitatea token-ului. Acest lucru se verific\u{a} folosid un obiect de tip AccessTokenTracker \c{s}i unul de tip ProfileTracker. Astfel \^{i}n cazul \^{i}n care se schimb\u{a} token-ul de acces \^{i}n timp ce aplica\c{t}ia ruleaz\u{a}, obiectul AccessTokenTracker va informa ProfileTracker-ul de schimbare, astfel, codul din metoda suprascris\u{a}, "onCurrentProfileChanged", actualizeaz\u{a} id-ul utilizatorului. Deci av\^{a}nd \^{i}n orice moment token-ul de acces actualizat, nu se vor sesiza schimb\u{a}rile din spate.
	
	Elementele de grafic\u{a} \c{s}i parametrii pentru butonul de autentificare se ini\c{t}ializeaz\u{a} \^{i}n metoda "onViewCreated". Se initializeaz\u{a} obiectele de tip ProfilePictureView \c{s}i TextView asociindu-le identificatorii din layout-ul de autentificare. Pentru butonul de autentificare, se seteaz\u{a} fragmentul curent, se seteaz\u{a} un vector de permisiuni \c{s}i se \^{i}nregistreaz\u{a} apelul cererii de autentificare. \^{I}n momentul \^{i}n care procesul de autentificare se finalizeaz\u{a}cu succes, se verific\u{a} \^{i}nc\u{a} odat\u{a} validitatea token-ul de acces \c{s}i se porne\c{s}te o nou\u{a} activitate, cea pentru lista de prieteni.

\subsection{List\u{a} de prieteni}

Prietenii care folosesc aplica\c{t}ia sunt afisa\c{t}i \^{i}ntr-o list\u{a} de tip RecyclerView \c{s}i este ob\c{t}inut\u{a} cu ajutorul unei interog\u{a}ri a platformei Facebook prin intermediul Graph API. \^{I}n metoda onCreate a activit\u{a}\c{t}ii FriendsActivity se a\c{s}teapt\u{a} ca aceasta s\u{a} fie invocat\u{a} prin intermediul unui obiect de tip Intent. Dac\u{a} activitatea este invocat\u{a}, se creaz\u{a} un obiect de tip GraphRequestBatch prin intermediul c\u{a}ruia se realizeaz\u{a} comunica\c{t}ia \^{i}ntre aplica\c{t}ie \c{s}i platforma Facebook. 

Cererea pentru lista de prieteni este implementat\u{a} \^{i}n metoda newMyFriendsRequest, pus\u{a} la dispozi\c{t}ie de c\u{a}tre Graph API. Pentru ca solicitarea s\u{a} fie valid\u{a} aceast\u{a} metod\u{a} prime\c{s}te ca parametru token-ul de acces al utilizatorului.

Informa\c{t}iile primite de la Facebook sunt \^{i}n format JSON \c{s}i con\c{t}in numele prietenilor \c{s}i identificatorii acestora. \^{I}n metoda onCompleted a cererii se parseaz\u{a} datele recep\c{t}ionate \c{s}i se populeaz\u{a} un obiect de tip List care va con\c{t}ine perechi "nume", "identificator". \^{I}n momentul \^{i}n care datele sunt gata de afis\c{a}t, se ini\c{t}ializeaz\u{a} obiectul RecyclerView cu lista de prieteni. 

\subsection{Grupuri}

Grupurile de Facebook sunt create pentru a da posibilitatea persoanelor care au acelea\c{s}i interese s\u{a} se reuneasc\u{a} \^{i}ntr-o comunitate restr\^{a}ns\u{a} de utilizatori. Acestea pun la dispozi\c{t}ie membrilor un spa\c{t}iu special de comunicare, care poate fi accesat doar de cei care fac parte din grup. Grupurile pot fi de trei feluri. Pot fi publice, adic\u{a} oricine poate deveni membru dac\u{a} face o cerere, la care poate s\u{a} r\u{a}spund\u{a} oricare membru sau dac\u{a} prime\c{s}te o invita\c{t}ie, trimis\u{a} tot de un membru oarecare. Con\c{t}inutul celor publice poate fi vizualizat \c{s}i de persoanele din afara grupului. Grupurile \^{i}nchise ofer\u{a} acelea\c{s}i privilegii ca \c{s}i cele publice, doar c\u{a}, informa\c{t}iile \^{i}mp\u{a}rt\u{a}\c{s}ite pe grup nu se pot citi din afara acestuia. Al treilea tip de grup este cel secret. \^{I}n grupurile secrete pot intra doar utilizatorii care primesc invita\c{t}ii de la cei care fac parte deja din grup. De asemenea con\c{t}inutul grupului nu poate fi v\u{a}zut \^{i}n afara lui. 

O alt\u{a} clasificare a grupurilor de Facebook este dup\u{a} modalitatea de creare. Ele pot fi create de aplica\c{t}ii care nu sunt dezvoltate de cei de la Facebook sau create de aplica\c{t}iile native ale platformei. Aplica\c{t}iile care nu sunt proprietatea platformei re\c{t}elei de socializare pot crea grupuri ale aplica\c{t}iei sau grupuri pentru jocuri. Membrii acestor grupuri sunt gestiona\c{t}i doar de aplica\c{t}ia care a creat grupurile. Astfel, \^{i}n aceast proiect, \^{i}n lista de grupuri afi\c{s}at\u{a} se reg\u{a}sesc doar acele grupuri care au fost create de aceast\u{a} aplica\c{t}ie. 

\subsubsection{Listarea grupurilor}
Pentru a ob\c{t}ine aceste grupuri se face o cerere prin intermediul Graph API, folosind un obiect de tip GraphRequest. Acesta trebuie s\u{a} primeasc\u{a} parametrii precum token-ul de acces al aplica\c{t}iei, metoda HTTP folosit\u{a}, \^{i}n acest caz HTTPMethod.GET, \c{s}i calea pentru informa\c{t}iile dorite. Calea este de forma "[identificatorul_aplica\c{t}iei]/groups". Pentru c\u{a} folosim token-ul de acces al aplica\c{t}iei, nodul din graful re\c{t}elei de socializare cerut este reprezentat de identificatorul aplica\c{t}iei, cel ob\c{t}inut folosind pa\c{s}ii din sec\c{t}iunea \ref{facebookSDK}. \^{I}n metoda onCompleted a cererii este primit\u{a} lista de grupuri sub form\u{a} de JSON. Aceste informa\c{t}ii sunt parsate \c{s}i este populat obiectul care re\c{t}ine grupurile sub form\u{a} de perechi identificator_grup, nume_grup. Tot aici, dupa ce datele sunt gata de afi\c{s}are, se actualizeaz\u{a} interfa\c{t}a grafic\u{a}.

\subsubsection{Crearea}

Pentru a crea un nou grup, se trimite o solicitare de tip POST tot cu ajutorul unui obiect GraphRequest, folosind metoda oferit\u{a} de Graph API, newPostRequest. Acestei metode i se paseaz\u{a} ca argumente, token-ul de acces al aplica\c{t}iei \c{s}i calea din graful re\c{t}elei. \c{S}i de aceast\u{a} dat\u{a} calea este de forma "[identificatorul_aplica\c{t}iei]/groups", dar cererii \^{i}i mai este asociate \c{s}i alte date, precum numele pe care o s\u{a} \^{i}l aib\u{a} noul grup, identificatorul administratorului, dac\u{a} se dore\c{s}te setat acest lucru \c{s}i informa\c{t}ii despre tipul grupului, cum ar fi tipul public. Datele primite \^{i}n urma acestei solicit\u{a}ri vin tot sub forma\u{a} de JSON \c{s}i con\c{t}in identificatorul noului grup creat. Dup\u{a} parsarea datelor, se actualizeaz\u{a} lista de grupuri din interfa\c{t}a grafic\u{a}.

\subsubsection{Ad\u{a}ugarea de membri}

Ad\u{a}ugarea de persoane \^{i}n grup se realizeaz\u{a} tot prin intermediul unei cereri, folosind Graph API. Pentru ca solicitarea \c{s}\u{a} fie valid\u{a}, metodei newPostRequest \^{i}i trebuie transmise token-ul de acces al aplica\c{t}iei \c{s}i calea din graful re\c{t}elei, care va fi un \c{s}ir de caractere de forma "[identificatorul_grupului]/members". Acestei  cereri, \^{i}i va fi ata\c{s}at parametrul "member" av\^{a}nd ca valoare identificatorul persoanei care urmeaz\u{a} s\u{a} fie ad\u{a}ugat\u{a} \^{i}n grup.

\subsubsection{\c{S}tergerea de membri}

\c{S}tergerea de persoane din grup se face asem\u{a}n\u{a}tor cu ad\u{a}ugarea lor. \c{S}i aici este nevoie de token-ul de acces al aplica\c{t}iei, iar calea din graful re\c{t}elei este la fel, adic\u{a} "[identificatorul_grupului]/members". Diferen\c{t}a vine din faptul c\u{a} este specificat\u{a} metoda HTTP folosit\u{a}. \^{I}n acest caz, HttpMethod.DELETE. Parametrul ata\c{s}at cererii este tot "member" \c{s}i are ca valoare tot identificatorul utilizatorului care urmeaz\u{a} s\u{a} fie \c{s}ters.

\subsubsection{\c{S}tergerea de grupuri}

Grupurile create de aplica\c{t}iei pot fi \c{s}terse \c{s}i ele, dar condi\c{t}ia este ca acestea s\u{a} nu mai aib\u{a} niciun membru. Astfel, dac\u{a} un utilizator dore\c{s}te s\u{a} \c{s}tearg\u{a} un grup, trebuie mai \^{i}nt\^{a}i s\u{a} elimine to\c{t}i membrii din acesta. Astfel printr-o cerere de tip GraphRequest, se ob\c{t}ine lista membrilor. Pentru aceast\u{a} solicitare, este nevoie de token-ul de acces al aplica\c{t}iei, iar calea din graful re\c{t}elei este "[identificator_grup]/members" \c{s}i de specificarea metodei folosite, adic\u{a} HttpMethod.GET. Lista este \^{i}ntoars\u{a} sub form\u{a} de JSON, este parsat\u{a} \c{s}i pentru fiecare membru se folose\c{s}te metoda de \c{s}tergere al unei persoane prezentat\u{a} mai sus. Dup\u{a} aceast\u{a} opera\c{t}ie se actualizeaz\u{a} \c{s}i lista de grupuri din interfa\c{t}a grafic\u{a}.

\subsubsection{Postarea de mesaje text}

Pentru postarea de mesaje text \^{i}n grup se folose\c{s}te din nou GraphRequest cu metoda newPostRequest, care va primi ca argumente, token-ul de acces al aplica\c{t}iei \c{s}i calea din graful re\c{t}elei de socializare, "[identificatorul_grupului]/feed". Acestei cereri \^{i}i este ata\c{s}at parametrul "message" care va avea ca valoare mesajul text care urmeaz\u{a} s\u{a} fie postat. Aceste post\u{a}ri apar pe grup, av\^{a}nd ca autor aceast\u{a} aplica\c{t}ie. 

\subsubsection{Listarea post\u{a}rilor}

Post\u{a}rile reg\u{a}site pe grupuri pot fi preluate \^{i}n aplica\c{t}ie folosind tot Graph API. Se trimite o solicitare de tip GraphRequest care prime\c{s}te token-ul de acces curent al aplica\c{t}iei,  calea din graful re\c{t}elei, "[identificatorul_grupului]/feed" \c{s}i metoda HTTP folosit\u{a}, HttpMethod.GET. Datele primite \^{i}n urma acestei cereri sunt sub form\u{a} de JSON. Acestea sunt parsate, c\u{a}ut\^{a}ndu-se dupa cuv\^{i}ntul cheie "message".

\subsection{Trimitere - Recep\c{t}ie mesaje}

Schimbul de mesaje folosind infrastructura Facebook, nu se mai poate implementa \^{i}n aplica\c{t}iile care folosesc instrumentele puse la dispozi\c{t}ie de aceast\u{a} platform\u{a}. Aceast\u{a} idee, de a bloca dezvoltatorii s\u{a} creeze aplica\c{t}ii pentru schimbul de mesaje prin intermediul re\c{t}elei de socializare, a fost adoptat\u{a} \^{i}n acest an, \^{i}ncep\^{a}nd cu 30 aprilie 2015. Decizia de a opri suportul pentru schimbul de mesaje private a fost luat\u{a} datorit\u{a} num\u{a}rului foarte mare de aplica\c{t}ii robot care nu f\u{a}ceau dec\^{a}t s\u{a} deranjeze prin trimiterea de mesaje. Astfel a trebuit s\u{a} g\u{a}sesc o modalitate de a evita folosirea SDK-ului de la Facebook.

Solu\c{t}ia g\u{a}sit\u{a} se bazeaz\u{a} pe gestionarea notific\u{a}rilor din sistemul de operare Android. Toate notific\u{a}rile primite pe un anumit sistem, pot fi interceptate de o aplica\c{t}ie Android. Acest lucru este posibil, doar dac\u{a} utilizatorul acord\u{a} privilegii de acces la notific\u{a}ri pentru aplica\c{t}ie. Pe l\^{a}ng\u{a} acest fapt, pentru o func\c{t}ionarea corect\u{a} este necesar\u{a} prezen\c{t}a aplica\c{t}iei native de Facebook Messenger pentru c\u{a} ini\c{t}ierea de conversa\c{t}ii nu se poate face f\u{a}r\u{a} aceasta.

\^{I}n momentul \^{i}n care un utilizator dore\c{s}te s\u{a} poarte o discu\c{t}ie privat\u{a} \c{s}i criptat\u{a} cu un alt utilizator, acesta trebuie s\u{a} ini\c{t}ieze conversa\c{t}ia din aplica\c{t}ia nativ\u{a} de Facebook Messenger. Astfel, utilizatorul de la cel\u{a}lalt cap\u{a}t al canalului de comunica\c{t}ie va primi o notificare de la "com.facebook.orca" care va con\c{t}ine mesajul pe care trebuia sa \^{i}l primeasc\u{a}. Av\^{a}nd aceast\u{a} notificare se poate r\u{a}spunde la mesaj pe baza datelor primite prin intermediul notific\u{a}rii. R\u{a}spunsul va fi primit tot sub form\u{a} de notificare, deci se poate trimite \^{i}napoi un mesaj partenerului de comunica\c{t}ie tot pe baza notific\u{a}rilor. Mesajul de ini\c{t}iere va fi \^{i}ntotdeauna necriptat \c{s}i va fi un mesaj de tip aten\c{t}ionare, a faptului c\u{a} cineva dore\c{s}te s\u{a} poarte o conversa\c{t}ie criptat\u{a}.

Notific\u{a}rile din sistemul de operare sunt interceptate prin intermediul unui serviciu, NotificationListenerService. Acesta implementeaz\u{a} o metod\u{a} de a extrage notific\u{a}ri dintr-un obiect de tip StatusBarNotification. Din acest obiect se poate ob\c{t}ine numele pachetului care a produs notificarea. \^{I}n acest proiect ne intereseaz\u{a} doar pachetul "com.facebook.orca". Acesta este folosit de Facebook la schimbul de mesaje private. 

Astfel, av\^{a}nd notificarea prezent\u{a} \^{i}n bara de stare a sistemului, putem crea un obiect de tip WearableExtender. Acesta este folosit pentru a extrage din notificare informa\c{t}iile suplimentare, cum ar fi, ac\c{t}iuni sau pagini existente \^{i}n notific\u{a}re. Notificarea preluat\u{a} din bara de ac\c{t}iuni con\c{t}ine un c\^{a}mp extra \^{i}n care reg\u{a}sim informa\c{t}ii precum mesajul text primit, numele transmi\c{t}\u{a}torului sau identificatorul notific\u{a}rii. Tot de aici putem extrage \c{s}i eticheta notific\u{a}rii sau con\c{t}inutul inten\c{t}iei care va trebui s\u{a} gestioneze datele recep\c{t}ionate.

\^{I}n momentul \^{i}n care o notificare apare \^{i}n bara de ac\c{t}iuni, aceasta se extrage \c{s}i se transmite mai departe cu ajutorul unui obiect EventBus. Acesta transmite informa\c{t}iile preluate, activit\u{a}\c{t}ii care gestioneaz\u{a} acest tip de evenimente. \^{I}n acest caz, activitatea se nume\c{s}te ChatActivity.

\^{I}n ChatActivity se implementeaz\u{a} metoda onEvent, care va executa codul doar atunci c\^{a}nd o notificare a sosit. Aici se verific\u{a} numele pachetului pe care \^{i}l folose\c{s}te notificarea. Acesta trebuie s\u{a} fie "com.facebook.orca" pentru ca ne intereseaz\u{a} doar datele primite de la Facebook Messenger. Deoarece, metoda onEvent ruleaz\u{a} \^{i}n fundal, pentru a actualiza interfa\c{t}a grafic\u{a} trebuie s\u{a} folosim metoda runOnUiThread pentru codul de actualizare. Mesajul primit se g\u{a}se\c{s}te \^{i}n c\^{a}mpul "android.text", iar numele transmi\c{t}\u{a}torului \^{i}n c\^{a}mpul "android.title".

Pentru a r\u{a}spunde unui mesaj, este nevoie de un obiect RemoteInput, ce ajut\u{a} la specificarea datelor care vor fi preluate de la aplica\c{t}ie \c{s}i pasate unei inten\c{t}ii. Deoarece, \^{i}n cazul acestui proiect nu se creaz\u{a} notific\u{a}ri, ci doar se r\u{a}spunde la ele pe baza datelor primite, nu trebuie s\u{a} construim noi obiecte de tip RemoteInput, pentru c\u{a} acestea vin odat\u{a} cu notific\u{a}rile. Trebuie doar modificate datele transmise inte\c{t}iei. 

Astfel, \^{i}n c\^{a}mpul extra al notific\u{a}rii se adaug\u{a} mesajul de r\u{a}spuns. C\^{a}nd r\u{a}spunsul este gata de transmis, se apeleaz\u{a} metoda addResultsToIntent care va primi ca argumente, obiectul RemoteInput modificat, o nou\u{a} inte\c{t}ie care va gestiona transmisia \c{s}i c\^{a}mpul extra care va con\c{t}ine mesajul. Dup\u{a} toate acestea, se poate trimite mesajul de r\u{a}spuns, folosind metoda send a inten\c{t}iei care gestioneaz\u{a} aceast\u{a} notificare.

Acest serviciu, este pornit, doar dac\u{a} aplica\c{t}ia este \^{i}n activitatea ChatActivity. Deci notific\u{a}rile sunt gestionate doar dac\u{a} ambii parteneri de conversa\c{t}ie au activat serviciul. \^{I}n metoda onResume a activit\u{a}\c{t}ii trebuie verificat dac\u{a} serviciul este activ sau nu. \^{I}n cazul \^{i}n care serviciul nu este activ, se \^{i}ncearc\u{a} activarea lui. \^{I}n cazul \^{i}n care activitatea este oprit\u{a}, o s\u{a} se opreas\c{a} \c{s}i serviciul. Acest lucru se face \^{i}n metoda onDestroy a activit\u{a}\c{t}ii.
 
\section{Integrarea cript\u{a}rii transmisiei de date}

Criptarea mesajelor transmise folose\c{s}te algoritmul AES, iar cheile sunt simetrice, adic\u{a} \c{s}i criptarea \c{s}i decriptarea este realizat\u{a} pe baza aceluia\c{s}i \c{s}ir de caractere. Pentru a ob\c{t}ine cheia secret\u{a} se utilizeaz\u{a} dou\u{a} \c{s}iruri de caractere. Unul ce reprezint\u{a} cheia \^{i}n clar \c{s}i unul care este folosit la ini\c{t}ializarea unui obiect de tip Cipher.

Bibliotecile Java pun la dispozi\c{t}ie metode de a implementa partea de criptare a datelor, u\c{s}ur\^{a}nd foarte mult munca depus\u{a} pentru acest proces. Astfel, se poate folosi un obiect de tip Cipher pentru ob\c{t}inerea unui cifru dintr-un \c{s}ir de octe\c{t}i. Acesta nu poate fi insta\c{t}iat direct, ci printr-o metod\u{a} numit\u{a} getInstance. Metoda prime\c{s}te ca argument un \c{s}ir de caractere de forma "algoritm/mod/padding", \^{i}n cazul acestei aplica\c{t}ii se folose\c{s}te "AES/CBC/PKCS5Padding". 

Algoritmul folosit este AES, care reprezint\u{a} un standard ce poate procesa blocuri de date de 128 de bi\c{t}i, folosind chei simetrice de tip cifru de lungime 128, 192 sau 256 de bi\c{t}i. Fiecare intrare \c{s}i ie\c{s}ire a acestui algoritm reprezint\u{a} o secven\c{t}\u{a} de 128 de bi\c{t}i de 1 sau 0.\cite{Miller:2009:AES:1823209}. Aceast\u{a} aplica\c{t}ie utilizeaz\u{a} o cheie simetric\u{a} de 128 de bi\c{t}i. 

Modul CBC, reprezint\u{a} \^{i}nl\u{a}n\c{t}uirea blocurilor de cifru. Pentru acest mod este necesar un vector de ini\c{t}ializare. \^{I}n criptarea care folose\c{s}te CBC, primul bloc este format prin aplicarea func\c{t}iei logice SAU-exclusiv \^{i}ntre primul bloc de text clar \c{s}i vectorul de ini\c{t}ializare. Primul bloc de text criptat este ob\c{t}inut prin aplicarea algoritmului de criptare pe blocul rezultat din procesul anterior. Urm\u{a}torul bloc este ob\c{t}inut prin aplicarea func\c{t}iei SAU-exclusiv \^{i}ntre primul bloc criptat \c{s}i urm\u{a}torul bloc de text \^{i}n clar \c{s}i a\c{s}a mai departe \cite{Dworkin:2001:SER:2206247}. Procesul de ob\c{t}inere a blocurilor este prezentat \^{i}n figura \ref{img:CBC-C}.

\fig[scale=0.6]{src/img/CBC-C.pdf}{img:CBC-C}{Criptarea folosind CBC}	
\newpage
Decriptarea \^{i}n modul CBC se face aplic\^{a}nd func\c{t}ia de decriptare a obiectului Cipher pe primul bloc de text criptat. \^{I}ntre blocul rezultat \c{s}i vectorul de ini\c{t}ializare se aplic\u{a} func\c{t}a logic\u{a} SAU-exclusiv, rezult\^{a}nd primul bloc de text clar. Urm\u{a}torul bloc este ob\c{t}inut prin aplicarea func\c{t}iei de decriptare pe al doilea bloc, iar \^{i}ntre rezultat \c{s}i primul bloc criptat se aplic\u{a} tot func\c{t}ia locic\u{a} SAU-exclusiv \cite{Dworkin:2001:SER:2206247}. Procesul de ob\c{t}inere a blocurilor de text \^{i}n clar este reprezentat \^{i}n figura \ref{img:CBC-D}.
  
\fig[scale=0.6]{src/img/CBC-D.pdf}{img:CBC-D}{Decriptarea folosind CBC}	

PKCS5Padding, reprezint\u{a} o schem\u{a} de formatare a unui text, astfel \^{i}nc\^{a}t lungimea acestuia s\u{a} fie multiplu de opt. Trebuie s\u{a} fie multiplu de opt, deoarece un bloc de text are lungimea de opt octe\c{t}i. Este necesar\u{a} aceast\u{a} metod\u{a} de completare a textului p\^{a}n\u{a} dimensiunea lui este multiplu de opt, deoarece se folose\c{s}te modul CBC, care se bazeaz\u{a} pe criptarea \^{i}nl\u{a}n\c{t}uit\u{a} a blocurilor de text.

Criptarea \c{s}i decriptarea unor date, presupune ca acestea s\u{a} fie reprezentate sub form\u{a} de octe\c{t}i, \^{i}n acest caz, vectori de octe\c{t}i. Deoarece informa\c{t}iile sunt transmise folosind protocolul HTTP, datele trimise sau recep\c{t}ionate sunt sub form\u{a} de \c{s}iruri de caractere. De aceea, \^{i}n acest\u{a} aplica\c{t}ie se va folosi codificarea Base64 a datelor ce va fi detaliat\u{a} \^{i}n continuare\cite{base64}.

\^{I}n codificarea Base64, 3 octe\c{t}i de intrare vor fi reprezenta\c{t}i prin 4 octe\c{t}i \cite{base644to3}. Dac\u{a} se dore\c{s}te codificarea unui vector de n octe\c{t}i va rezulta un \c{s}ir de caractere a c\u{a}rui dimensiunea este reprezentat\u{a} \^{i}n urm\u{a}toare schem\u{a}: 
 \[ octe\text{\c{t}}i[n] \rightarrow caractere\Big[\ceil[\Big]{\frac{4 * n}{3}}\Big] \]
 
 De exemplu, pentru n = 32, rezultatul este 42.6, care este rotunjit superior la cel mai apropiat multiplu de 4, deci \c{s}irul de caractere va avea lungimea de 44 de octe\c{t}i.
\subsection{Criptarea}

Pentru criptare, se creaz\u{a} un obiect de tip Cipher prin metoda getInstance care prime\c{s}te ca argument \c{s}irul de caractere "AES/CBC/PKCS5Padding", dup\u{a} cum se poate observa \^{i}n sec\c{t}iunea de cod \ref{lst:criptare}. Se instan\c{t}iaz\u{a} un obiect SecretKeySpec din cheia de criptare. \c{S}irurile de caractere nu pot fi procesate direct, ele trebuie convertite la un vector de octe\c{t}i, acest lucru f\u{a}c\^{a}ndu-se cu ajutorul metodei getBytes, c\u{a}ruia se specific\u{a} setul de caractere, \^{i}n acest caz "UTF-8".

\lstset{caption=Cod folosit pentru criptare,label=lst:criptare}
\lstinputlisting{src/code/build/criptare.class}

Cifrul trebuie ini\c{t}ializat, acest lucru realiz\^{a}ndu-se prin metoda init. Aceasta prime\c{s}te ca argumente, modul de ini\c{t}ializare, aici ENCRYPT_MODE, obiectul SecretKeySpec \c{s}i un obiect de tipul IvParameterSpec ce reprezint\u{a} vectorul de ini\c{t}ializare. Este nevoie de acest vector pentru c\u{a} se folose\c{s}te modul CBC, care necesit\u{a} un astfel de obiect pentru a procesa primul bloc din text. \^{I}n aceast\u{a} aplica\c{t}ie vectorul de ini\c{t}ializare are valoarea "AAAAAAAAAAAAAAAA", dar poate avea oricare alt\u{a} valoare.

Rezultatul cript\u{a}rii este un vector de octe\c{t}i, returnat de metoda doFinal a cifrului. Aceast\u{a} metoda prime\c{s}te ca argument textul care urmeaz\u{a} a fi criptat, dar sub form\u{a} de vector de octe\c{t}i. Pentru a ob\c{t}ine un format de afi\c{s}are a vectorului de octe\c{t}i se folose\c{s}te codificare Base64.

\subsection{Decriptarea}

Pentru ca decriptarea s\u{a} func\c{t}ioneze, trebuie s\u{a} se primeasc\u{a} un \c{s}ir de caractere care reprezint\u{a} criptarea unui text clar, dar folosind aceea\c{s}i cheie de criptare, acela\c{s}i algoritm \c{s}i aceea\c{s} metod\u{a} de codificare, adic\u{a} Base64. Codul folosit pentru decriptare este prezentat \^{i}n sec\c{t}iunea de cod \ref{lst:decriptare}.

\lstset{caption=Cod folosit pentru decriptare,
label=lst:decriptare}
\lstinputlisting{src/code/build/decriptare.class} 

Aplica\c{t}ia prime\c{s}te un \c{s}ir de caractere ce trebuie decodificat \c{s}i convertit la un vector de octe\c{t}i prin metoda Base64, folosind setul de caractere UTF-8. Dup\u{a} aceast\u{a} decodificare se creaz\u{a} obiectul de tip Cipher tot prin getInstance("AES/CBC/PKCS5Padding"). \c{S}i aici este nevoie de  SecretKeySpec, format pe baza aceleia\c{s}i chei, ca la criptare. Se ini\c{t}ializeaz\u{a} cifrul, la fel ca la partea de criptare, doar ca aici metoda specificat\u{a} este DECRYPT_MODE.

Rezultatul este \^{i}ntors tot sub form\u{a} de vector de octe\c{t}i, deci este necesar\u{a} conversia la \c{s}ir de caractere, pentru a putea fi afi\c{s}at \^{i}n interfa\c{t}a grafic\u{a}.

 