\chapter{Proiectarea arhitecturii}

Aplica\c{t}ia \^{i}\c{s}i propune s\u{a} \^{i}mbun\u{a}t\u{a}\c{t}easc\u{a} securitatea conversa\c{t}iilor din cadrul re\c{t}elelor de socializare, precum Facebook sau Google Plus. Mai exact este vorba de criptarea mesajelor private transmise prin intermediul unei re\c{t}ele de socializare, dar \c{s}i a post\u{a}rilor din anumite spa\c{t}ii private de comunicare, precum grupurile create de aplica\c{t}ie.

Pentru asigurarea securit\u{a}\c{t}ii, fiecare dispozitiv mobil care utilizeaz\u{a} aplica\c{t}ia este considerat un echipament de \^{i}ncredere ce va proteja prin criptare mesajele, \^{i}nainte de a le trimite c\u{a}tre platforma de socializare.

Cheile folosite pentru criptare sunt comunicate \^{i}ntre participan\c{t}i independent de re\c{t}eaua social\u{a}, folosind alte c\u{a}i de comunicare, precum SMS, coduri QR sau direct prin telefon, a\c{s}a cum se poate vedea \c{s}i \^{i}n Figura \ref{img:App-Arh}.
\fig[scale=0.5]{src/img/App-Arh.pdf}{img:App-Arh}{Arhitectură}

\section{Conectarea la re\c{t}eaua social\u{a}}

Folosirea aplica\c{t}iei presupune ca utilizatorul s\u{a} de\c{t}in\u{a} un cont pe re\c{t}eaua de socializare, cu ale c\u{a}rui creden\c{t}iale se poate realiza autentificarea, a\c{s}a cum este prezentat \c{s}i \^{i}n Figura \ref{img:Login}. \^{I}n cazul \^{i}n care contul este setat deja \^{i}n sistemul de operare, acesta se preia automat de c\u{a}tre aplica\c{t}ie \c{s}i se autentific\u{a} utilizatorul. De asemenea, este necesar ca dispozitivul s\u{a} aib\u{a} instalat\u{a} aplica\c{t}ia nativ\u{a} pus\u{a} la dispozi\c{t}ie de platforma re\c{t}elei.

\fig[scale=0.7]{src/img/Login.pdf}{img:Login}{Autentificare}

Autentificare printr-o alt\u{a} aplica\c{t}ie dec\^{a}t cea nativ\u{a} necesit\u{a} permisiunea utilizatorului ca aplica\c{t}ia s\u{a} acceseze datele din contul s\u{a}u. Dac\u{a} utilizatorul este de acord cu accesarea datelor de c\u{a}tre aplica\c{t}ie, \^{i}nseamn\u{a} c\u{a} procesul s-a terminat cu succes. Dac\u{a} nu, utilizatorul este obligat s\u{a} re\^{i}nceap\u{a} procesul de autentificare.

Dac\u{a} autentificare s-a \^{i}ncheiat cu succes, aplica\c{t}ia prime\c{s}te din partea platformei re\c{t}elei sociale un token de acces, care a fost decris \^{i}n sec\c{t}iunea \ref{facebookSDK}. Cu acesta se vor deschide noi sesiuni de comunica\c{t}ie \^{i}ntre aplica\c{t}ia client \c{s}i re\c{t}eaua de socializare.

\section{Conversa\c{t}ii criptate}

Pa\c{s}ii ce trebuie urma\c{t}i pentru a purta o conversa\c{t}ie criptat\u{a} sunt prezenta\c{t}i \^{i}n Figura \ref{img:ConversatiiCriptate} \c{s}i \^{i}i vom descrie \^{i}n continuare:

\fig[scale=0.6]{src/img/ConversatiiCriptate.pdf}{img:ConversatiiCriptate}{Schimbul de mesaje}

\paragraph{Pasul 1:}

Ini\c{t}ierea conversa\c{t}iilor necesit\u{a} aplica\c{t}ia nativ\u{a} de Messenger a re\c{t}elei sociale. Mesajul de ini\c{t}iere este trimis \^{i}n clar prin intermediul aplica\c{t}iei native \c{s}i este de preferat s\u{a} nu reprezinte informa\c{t}ii relevante pentru cineva. Acesta poate fi de exemplu \c{s}irul de caractere "Hello". La receptor se va primi o notificare ce va con\c{t}ine mesajul de ini\c{t}iere, notificare pe baza c\u{a}reia se va putea r\u{a}spunde la acest mesaj.

\paragraph{Pasul 2:}

Schimbul de chei se poate face \c{s}i \^{i}nainte de mesajul de ini\c{t}ializare dar \c{s}i dup\u{a} acesta. Pentru a decripta cu succes mesajele, cei care particip\u{a} la conversa\c{t}ie trebuie s\u{a} de\c{t}in\u{a} cheile de criptare ale partenerilor. Schimbul de chei va fi prezentat mai detaliat \^{i}n sec\c{t}iunea \ref{chei}. 

\paragraph{Pasul 3:}

\^{I}n momentul \^{i}n care mesajul de ini\c{t}iere a conversa\c{t}iei ajunge la destinatar, se \c{s}tie c\u{a} se poate \^{i}ncepe schimbul de mesaje criptate folosind cheile de criptare propuse de fiecare participant. Acum receptorul poate s\u{a} r\u{a}spund\u{a} la notificarea primit\u{a} printr-un mesaj protejat.

\paragraph{Pasul n:}

Cheile se pot modifica pe parcursul unei conversa\c{t}ii. Deci f\u{a}r\u{a} o gestiune avansat\u{a} de chei, mesajele schimbate nu vor mai putea fi decriptate \^{i}n viitor, pentru c\u{a} nu se vor mai \c{s}ti cheile de criptate pentru mesaje. De aceea participan\c{t}ii la conversa\c{t}ii pot folosi o singur\u{a} cheie pentru o perioad\u{a} mai lung\u{a} de timp, \^{i}n ideea c\u{a} se va dori decriptarea informa\c{t}iilor.

\^{I}n locul aplica\c{t}iilor native de Messenger, se va folosi un serviciu care intercepteaz\u{a} notific\u{a}rile primite \^{i}n sistemul de operare. Vor fi selectate notific\u{a}rile care vin de la re\c{t}eaua de socializare \c{s}i cele care vin de la prietenul selectat din lista de prieteni care folosesc aplica\c{t}ia.

\section{Grupuri protejate}

Dup\u{a} procesul de autentificare, utilizatorul poate beneficia de func\c{t}ionalit\u{a}\c{t}ile disponibile. Pentru \^{i}nceput se va afi\c{s}a lista de prieteni care, de asemenea folosesc aceast\u{a} aplica\c{t}ie. \^{I}n continuare vom descrie fluxul de ac\c{t}iuni prezente \^{i}n Figura \ref{img:Groups}.

\fig[scale=0.5]{src/img/Gestionare-Grupuri.pdf}{img:Groups}{Gestionare grupuri protejate}

Prin selectarea unui prieten, se va deschide o nou\u{a} activitate de unde se pot ini\c{t}ia conversa\c{t}ii sau gestiona grupuri.

Grupurile create de aplica\c{t}ie sunt afi\c{s}ate \^{i}ntr-o list\u{a}. Se ofer\u{a} posibilitatea de a selecta un grup \c{s}i de a aplica pe el o operta\c{t}ie de \c{s}tergere grup, care va avea ca efect eliminarea din list\u{a} a grupului sau ad\u{a}ugare \c{s}i eliminare membrii din grup.

De asemenea se pot crea noi grupuri ale aplica\c{t}iei sau se pot vedea mesajele postate pe un grup selectat. Mesajele care sunt postate din aplica\c{t}ie sunt criptate.

Pentru vizualizarea sau postarea mesajelor se apas\u{a} butonul de deschidere a grupului \c{s}i este nevoie ca utilizatorul s\u{a} cunoasc\u{a} o cheie de criptare. Aceasta va fi prezentat\u{a} \^{i}n sec\c{t}iunea \ref{chei} \c{s}i poate fi diferit\u{a} de la grup la grup, astfel este sporit nivelul de securitate. Dup\u{a} ce mesajul a fost postat cu succes, acesta apare \^{i}n lista de post\u{a}ri.

Dac\u{a} cineva posteaz\u{a} mesaje pe un grup al aceste aplica\c{t}ii, dar folosind un browser sau aplica\c{t}iile native, acele mesaje nu vor putea fi afi\c{s}ate folosind aceast\u{a} aplica\c{t}ie, deoarece decriptarea aplicat\u{a} pe un mesaj \^{i}n clar nu va func\c{t}iona.

Pentru gestionarea grupurilor se folose\c{s}te SKD-ul pus la dispozi\c{t}ie de re\c{t}eaua de socializare. Toate cererile sunt trimise prin intermediul Graph API-ului, prezentat \^{i}n sec\c{t}iunea \ref{graphapi}.

\section{Partajare chei}
\label{chei}

Cheile folosite \^{i}n aceasta aplica\c{t}ie sunt simetrice, deci utilizatorii trebuie s\u{a} de\c{t}in\u{a} acelea\c{s}i chei pentru a putea avea o conversa\c{t}ie criptat\u{a} sau pentru a putea posta sau vizualiza mesaje de pe grupuri.

Faptul c\u{a} se \^{i}ncearc\u{a} securizarea unor canale de comunica\c{t}ie nesigure duce la problema schimbului de chei. Este de evitat trimiterea cheilor pe aceste canale nesigure, de aceea vor fi folosite alte modalit\u{a}\c{t}i de comunicare cum ar fi SMS sau coduri QR.

\^{I}n momentul \^{i}n care se dore\c{s}te \^{i}nceperea unei conversa\c{t}ii, partenerii de dialog trebuie s\u{a} cad\u{a} de acord asupra unei chei de criptare. Ace\c{s}tia \^{i}\c{s}i pot comunica cheia prin SMS sau prin intermediul unui cod QR. Pentru schimbul de mesaje, cheile se pot schimba de c\^{a}te ori doresc participan\c{t}ii.

Pentru a posta mesaje criptate sau pentru a citi \c{s}i decripta mesaje din cadrul unui grup protejat, este nevoie ca cei care doresc s\u{a} participe cu informa\c{t}ii s\u{a} afle cheia de criptare. Cheia se transmite tot prin SMS sau coduri QR, dar este transmis\u{a} doar de persoanele care fac parte din grup. Astfel dac\u{a} unui membru i se cere cheia de criptare, acesta poate evalua dac\u{a} persoana care vine cu aceast\u{a} solicitare este \c{s}i ea membru al grupului \c{s}i dac\u{a} este de \^{i}ncredere.